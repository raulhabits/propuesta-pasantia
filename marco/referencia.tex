\begin{center}
\section{MARCO DE REFERENCIA}
\end{center}

\subsection{MARCO TE\'ORICO}

\subsubsection{Algoritmo}

\noindent
Es una sucesi\'on finita de instrucciones diseñadas para un prop\'osito espec\'ifico \cite{Algoritmo1}, adem\'as contiene una descripci\'on de los datos para resolver el proposito en menci\'on \cite{Algoritmo2}.

\subsubsection{Python}

\noindent
Es un lenguaje de programaci\'on de proposito general, muy popular en el \'area cientifica que permite incluir c\'odigo escrito en otros lenguajes. Adem\'as al ser un lenguaje interpretado, requiere un interprete para ser traducido a lenguaje maquina por lo tanto puede ejecutarse en diferentes sistemas operativos \cite{Python}.

\subsubsection{R}

\noindent
Es un paquete estad\'istico muy popular en la comunidad cient\'ifica para la manipulaci\'on y tratamiento de datos. Adem\'as cuenta con una variedad de extensiones que permiten realizar an\'alisis especializado de informaci\'on, ya sea desde un punto de vista gr\'afico, hasta modelos de regresi\'on, entre otros \cite{R}.

\subsubsection{Estad\'istica Descriptiva}

\noindent
Es el conjunto de t\'ecnicas para la presentaci\'on y reducci\'on de los datos, ya sea desde el punto de vista gr\'afico, a trav\'es de la obtencion de las variables estad\'isticas (moda, media, mediana, varianza, entre otros), e inclusive utilizando t\'ecnicas que estudian la dependencia entre dos o mas caracter\'isticas (regresi\'on y correlaci\'on) \cite{EstadisticaDescriptiva}.

\subsubsection{M\'odelo de Regresi\'on lineal}

\noindent
Cuando se puede determinar que existe una correlaci\'on entre dos variables, es decir que una variable pueda describir el comportamiento de otra el m\'odelo de regresi\'on lineal consiste en una prueba de hip\'otesis para determinar si existe una recta cuyo coeficiente de error sea aceptable. Aunque este m\'odelo no esta limitado a dos variables, ya que se puede extender al modelo de regresi\'on lineal multivariado \cite{ModeloRegresionLineal}.


\subsubsection{Aprendizaje Supervisado}

\noindent
Es cuando en el proceso de aprendizaje hay un maestro o supervisor, que se encarga de alimentar a por medio de datos marcados, validar las respuestas esperadas y hacer ajustes al modelo \cite{Aprendizaje}.

\subsubsection{Aprendizaje No Supervisado}

\noindent
Es cuando en el proceso de aprendizaje no hay un maestro o supervisor, es decir el modelo se encarga de identificar patrones, hacer correlaciones, y agrupar caracter\'isticas que arrojen similitudes en la variable de inter\'es \cite{Aprendizaje}.


\subsubsection{Redes Neuronales}

\noindent
Son m\'odelos matem\'aticos que teorizan el comportamiento del cerebro, explorando y reproduciendo informaci\'on de una forma similar al cerebro. Su uso frecuentemente es el an\'alisis de datos, detecci\'on de patrones, entre otros \cite{RedesNeuronales}.

\subsubsection{Redes Neuronales Convolucionales}

\noindent
Es un caso especial de las redes neuronales, en el cual se asume que las entradas son im\'agenes o estan codificadas como si lo fuesen, y las neuronas est\'an marcadas por pesos permitiendo resaltar caracter\'isticas \cite{RedesNeuronalesConvolucionales}.



\subsection{MARCO TEMPORAL}
\noindent El proyecto será desarrollado durante 5 meses a partir de septiembre del año 2020 hasta enero del año 2021.