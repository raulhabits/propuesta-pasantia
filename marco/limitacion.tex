\begin{center}
\section{ALCANCE Y LIMITACIONES}
\end{center}

\subsection{ALCANCE}

\noindent Este proyecto pretende aportar una soluci\'on de software que se constituya como una fuente de datos confiable, importante y útil para la administración de cultivos en ambientes controlados. Por lo tanto es pertinente aclarar y resaltar que:

\begin{itemize}
	\item Para el desarrollo del prototipo no se incluir\'a la publicaci\'on en tiendas como lo son \textit{Play Store y App Store}.
	\item En condiciones id\'oneas sería de gran valor tener todas las variables ambientales que interactuan directa e indirectamente sobre el objeto de estudio, pero teniendo en cuenta que tanto el rendimiento de la aplicaci\'on  y los costos de infraestructura se podrían ver afectados considerablemente, se recolectara unicamente la informaci\'on que provea alguno de los dispositivos dedicados a esa labor y que est\'en disponibles en el mercado. 
	\item No se har\'a ningún tipo de an\'alisis, clasificación o tratamiento a los datos recolectados ya que el prototipo se limita a la presentaci\'on de la informaci\'on y \'unicamente se realizaran las transformaciones respectivas para cumplir con ese fin.
\end{itemize}

%\lipsum[5]
\subsection{LIMITACIONES}
%\lipsum[6]

\noindent Entre los diversos factores que pueden presentarse en el desarrollo y ejecución de una solución tecnológica; las siguientes son de gran relevancia y alto impacto para la ejecución de este proyecto, por lo tanto es necesario considerar:

\begin{itemize}
	\item \textit{La cobertura y velocidad del canal de comunicaciones.} Teniendo en cuenta que si falla o no funciona adecuadamente; el producto de software no tendrá el resultado y/o el impacto esperado.
	\item \textit{La capacidad de almacenamiento y procesamiento los dispositivos es limitada.} Por lo tanto no se guardaran registros ambientales en la memoria del dispositivo donde se instale el prototipo y la aplicaci\'on sera compatible con los tel\'efonos que se determinen en la fase de diseño e implementaci\'on.
\end{itemize}