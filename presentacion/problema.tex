\begin{center}
	\section{PROBLEMA DEL ENTORNO ACAD\'EMICO} 
	%\vspace{0.5mm} \bigrule
	\addcontentsline{toc}{section}{PROBLEMA DEL ENTORNO ACAD\'EMICO}
\end{center}


\noindent
En consecuencia de la evoluci\'on de las tecnolog\'ias de informaci\'on y comunicaciones diversas pr\'acticas y procesos se han adaptado a nivel industrial, social, cultural, entre otros. De los cuales se puede resaltar la transmisi\'on del conocimiento por medio de canales virtuales redefiniendo la forma y el alcance en que se comparte la informaci\'on haciendo uso de diferentes tecnolog\'ias y herramientas de comunicaci\'on \cite{PsicologiaEducacionVirtual2009}. 
\\

\noindent
A pesar que hay varias corrientes en cuanto a la percepci\'on del t\'ermino educaci\'on virtual donde en ocasiones se le confunde con la sustituci\'on de un libro impreso y un aula de clase por un recurso electr\'onico. Hay quienes la interpretan como una alternativa en el camino de la formaci\'on ya que conlleva los siguientes desaf\'ios \cite{EducacionVirtualManuelUnigarro}.
\\
\begin{itemize}
	\item Contexto
	\item Definici\'on
	\item Metodolog\'ia
	\item Contenido
	\item Herramientas
	\item Expectativas
\end{itemize}

\noindent
Teniendo en cuenta que la ciencia de los datos \textit{(Data Science)} es un \'area de estudio del grupo de investigaci\'on e inteligencia computacional \textbf{\textit{LASER}} \cite{GrupoDeInvestigacionLaser} y la adopci\'on el uso de las tecnolog\'ias de la informaci\'on y comunicaciones. El grupo de investigaci\'on necesita generar un espacio virtual de apoyo a los estudiantes por medio de material did\'actico y ejercicios reproducibles en temas de estad\'istica, machine learning e inteligencia artificial.



\subsection{PLANTEAMIENTO DEL PROBLEMA}

\noindent
En el mercado actualmente hay varias herramientas de formaci\'on virtual con diferentes enfoques de las que se destacan:

\begin{enumerate}
	\item \textbf{\textit{\href{https://www.coursera.org}{Coursera}}} Plataforma con cursos en v\'ideo ofrecidos por m\'as de 115 universidades e instituciones educativas del mundo \cite{Coursera}.
	\item \textbf{\textit{\href{https://www.udemy.com}{Udemy}}} Plataforma con más de 150000 cursos en v\'ideos hechos por particulares sobre diversos temas entre ellos el aprendizaje de las máquinas \textit{machine learning}, inteligencia artificial y blockchain \cite{Udemy}. 
	\item \textbf{\textit{\href{https://jupyter.org}{Jupyter}}} Es una aplicaci\'on web de c\'odigo abierto para la ejecuci\'on de secuencias de c\'odigo en alrededor de 40 lenguajes de programacion y soporta  tecnolog\'ias relacionadas con \textit{Big Data} \cite{Jupyter}.
	\item \textbf{\textit{\href{https://colab.research.google.com/notebooks/intro.ipynb}{Google Colab}}} Es una aplicaci\'on web de Google similar a Jupyter pero solo admite el uso del lenguaje de programaci\'on Python, no requiere configuración para usarlo y brinda acceso gratuito a recursos computacionales, incluidas GPU, pero tiene algunas restricciones excepto cuando se utiliza la versi\'on de pago \cite{GoogleColab}.
\end{enumerate}

\noindent
Esta pasant\'ia puede convertirse en una aproximaci\'on al proyecto del grupo de investigaci\'on asociado a la construcci\'on de un espacio virtual de apoyo a los estudiantes en el \'area de ciencia de los datos utilizando las tecnolog\'ias consultadas.

\subsection{FORMULACIÓN DE PREGUNTA COMO SOLUCIÓN PROPUESTA}
%\lipsum[4-8] \index{Fomulación}

\noindent ¿C\'omo construir un espacio virtual que permita a los estudiantes la interacci\'on con diferentes pr\'acticas dedicadas al an\'alisis y tratamiento de informaci\'on?

\subsection{SISTEMATIZACI\'ON DEL PROBLEMA}

\begin{enumerate}
	\item ¿Qu\'e conjunto de tecnolog\'ias se van a utilizar?
	\item ¿Cu\'ales son los conjuntos de datos que se van a utilizar?
	\item ¿Qu\'e tratamiento se le dar\'a a los conjuntos de datos?
	\item ¿Qu\'e algoritmos se van a utilizar?
	\item ¿C\'omo ser\'a la interacci\'on de los estudiantes con el material?
\end{enumerate}





