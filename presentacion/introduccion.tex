
%\section*{\begin{center}INTRODUCCIÓN\end{center}}
%\addcontentsline{toc}{section}{INTRODUCCIÓN}
\begin{center}
\section*{INTRODUCCIÓN}
\addcontentsline{toc}{section}{INTRODUCCIÓN}
\end{center}


\noindent La agricultura ha tenido un papel fundamental en el desarrollo del hombre, la econom\'ia, y la sociedad. Por ejemplo, Friedrich Engels resalta la evoluci\'on del ser humano desde su etapa primitiva ya que dicha evoluci\'on conlleva a cambios en los medios de producci\'on; como es el caso de la agricultura y menciona que:\\

\begin{quotation}
	``El trabajo es la fuente de toda riqueza, afirman los especialistas en Economía pol\'itica. \textit{Lo es, en efecto, a la par que la naturaleza, proveedora de los materiales que \'el convierte en riqueza.} Pero el trabajo es much\'isimo m\'as que eso. Es la condición b\'asica y fundamental de toda la vida humana.''
	\cite{FriedrichEngels1876} 
\end{quotation}

\noindent Por ello es relevante no desconocer que el perfeccionamiento y evoluci\'on de la agricultura; no solo incide directamente en la econom\'ia y la calidad de vida, tambi\'en contribuye al crecimiento de diferentes \'areas tales como en la ciencia, industria, cultura entre otros. Un claro ejemplo de este fen\'omeno en un contexto hist\'orico de gran importancia y trascendencia para la humanidad se puede constatar  en la revoluci\'on industrial. \cite{KrismarEducacion2017} \\

\noindent Un siglo despu\'es a la reflexi\'on de Engels, Jos\'e Mar\'ia Figueres menciona que \textit{``no nos preocupamos por ella porque la gozamos''} \cite{JoseMariaFigueres1998} al hacer una comparaci\'on muy particular entre la agricultura, y la salud ya que propone la siguiente reflexi\'on:

\begin{quotation}
	``Cuando comprendemos el desaf\'io que significa duplicar la producci\'on mundial de alimentos en los pr\'oximos 20 años y hacerlo sobre la misma base de recursos naturales que hoy disponemos, comprendemos un poco que podemos "perder la salud" si no invertimos lo suficiente en tecnolog\'ia e infraestructura de producci\'on.'' \cite{JoseMariaFigueres1998}
\end{quotation}

\noindent En resumen de lo anterior, la evoluci\'on del ser humano ha sido acompañada por el uso de t\'ecnicas m\'as sofisticadas y eficientes para afrontar los diversos desaf\'ios en cuanto a la preservaci\'on y/o mejora en la calidad de vida. Por lo tanto, se introduce el concepto de Agricultura Inteligente derivado de la combinaci\'on de las  tecnolog\'ias de la informaci\'on y comunicaciones, el Internet de las Cosas, el an\'alisis, el desarrollo de aplicaciones m\'oviles y el procesamiento de grandes vol\'umenes de datos como lo sugieren los trabajos \cite{SmartFarmingTechnologies2019}
 \cite{IoTBasedSmartAgriculture2016} \cite{IoTBasedSmartAgricultureMonitoringSystem2017} enfocados al desarrollo de soluciones de ingenier\'ia para abordar las necesidades de la industria agr\'icola.\\

\noindent En conclusi\'on, la propuesta presentada en este documento esta enfocada en el diseño y construcci\'on del prototipo de una aplicaci\'on m\'ovil para la realizaci\'on de tareas asociadas a la recolecci\'on y seguimiento de la informaci\'on ambiental orientada al cuidado de las plantas y/o cultivos agr\'icolas.



